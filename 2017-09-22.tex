\documentclass{article}
\usepackage{amsmath}
\usepackage{amssymb}
\usepackage{titlesec}

\title{Homework}
\date{2017-09-22}
\author{Ding Yaoyao, 516030910572}

\begin{document}

	\maketitle

	\section*{Exercise 1-3}
		\subsection*{5}

		\begin{itemize}
			\item Closure: If $a,b \in G$, there are $c,d \in G$ such that $a = c^m, b = d^m$. Then $ab=c^md^m=(cd)^m \in G$(Beacuse group $G$ is an abelian group).
			\item Identity: $e = e^m \in G$.
			\item Inverse: If $a = b^m \in G$, then $a^{-1} = (b^m)^{-1} = (b^{-1})^m \in G$.
		\end{itemize}

		Above all, $H \preccurlyeq G$.

		\subsection{6}

		For all $a, b \in gHg^{-1}$, there are $c,d \in H$ such that $a = gcg^{-1}, b = gdg^{-1}$. 
		
		Then $a^{-1}b = (gcg^{-1})^{-1}gdg^{-1} = gc^{-1}g^{-1}gdg^{-1} = gc^{-1}dg^{-1}$. 
		
		Beacuse $c^{-1}d \in H$, we have $a^{-1}b \in gHg^{-1}$. 
		
		Finally, we get $gHg^{-1} \preccurlyeq G$.

		\subsection*{7}

		For all $b, c \in C(a)$, we have $ba = ab$ and $ca = ac$.

		Then $b^{-1}c = (aba^{-1})^{-1}aca^{-1} = ab^{-1}a^{-1}aca^{-1} = ab^{-1}ca^{-1}$, which imply that $ab^{-1}c=b^{-1}ca$, $b^{-1} \in C(a)$.

		Above all, $C(a) \preccurlyeq G$.

		\subsection*{8}

		

		\subsection*{18}
		\subsection*{19}

\end{document}

