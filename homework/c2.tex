\documentclass[draft]{article}
\usepackage{amsmath}
\usepackage{amssymb}
\usepackage{titlesec}
\usepackage{booktabs}

\title{Homework}
\date{2017-11}
\author{Ding Yaoyao, 516030910572}

\begin{document}

	\maketitle

	\section*{Exercise 2-1}
		\subsection*{1}
			The $3$ left cosets of $H$ in $A_4$ are:
			$$
				(1),(12)(34),(13)(24),(14)(23)
			$$
			$$
				(123),(134),(243),(142)
			$$
			$$
				(234),(132),(143),(124)
			$$
			The $6$ left cosets of $H$ in $S_4$ are:
			$$
				(1),(12)(34),(13)(24),(14)(23)
			$$
			$$
				(123),(134),(243),(142)
			$$
			$$
				(234),(132),(143),(124)
			$$
			$$
				(12),(34),(1324),(1423)
			$$
			$$
				(13),(24),(1234),(1432)
			$$
			$$
				(14),(23),(1243),(1342)
			$$
		\subsection*{8}
			There are 2 left cosets of $\langle a^4 \rangle$ and they are:
			$$
				\langle a^2 \rangle
			$$
			and
			$$
				\{a, a^3, a^5, \cdots, a^{29}\}
			$$
		\subsection*{11}
			Let $G$ be a group and $H \preceq G$. Then $aH$ is a left coset of $H$
			generated by $a \in G$. We have:
			$$
				(aH)^{-1} = H^{-1}a^{-1} = Ha^{-1}
			$$
			So $(aH)^{-1}$ is a right coset of $H$ generated by $a^{-1}$.
		\subsection*{12}
			\begin{equation*}
				\begin{split}
					x \in a(H_1\cap H_2) \Leftrightarrow x = ah(h \in H_1\;\text{and}\;h \in H_2) \\
					\Leftrightarrow x \in aH_1\;\text{and}\;x \in aH_2 \Leftrightarrow x \in aH_1 \cap aH_2
				\end{split}
			\end{equation*}
			So $a(H_1\cap H_2) = aH_1 \cap aH_2$.
		\subsection*{20}
			The factors of $33$ are $1, 3, 11, 33$, so the order of every element in
			$G$ must be one of them. For any $g \in G$, if $ord(g) = 3$ or $ord(g) =
			33$ then we find the element we want(it is $g$ or $g^{11}$). So we only 
			need to prove that such group $G$ does not exist:
			$$
				ord(g) = 11 \quad \text{for any non-unit} \; g \in G 
			$$
			Let $S_1, S_1, \dots, S_k$ be all the different cyclic subgroup of $G$.
			Beacuse every element in $G$ can generate a cyclic subgroup, we have $G =
			\bigcup S_i$.  For all $i \neq j$, $S_i \cap S_j = \{e\}$(If $g \in S_i \cap S_j$ and $g
			\neq e$, we have $S_i = S_j = \langle g \rangle$).There is only one cyclic
			subgroup contains $1$ element which is $\{e\}$ and other cyclic subgroup has
			$11$ elements in it.So we have :
			$$
			\vert G \vert = 1 + 10(k - 1) = 33 \quad (k \in \mathbb{Z})
			$$
			Such $k$ does not exist, so such group $G$ does not exist.
			Above all, every group $G$ follow $\vert G \vert = 33$ has an element whose
			order is $3$.
		\subsection*{22}
			Let $\phi:G \rightarrow G(a \rightarrow a^n)$ be the map($G$ is abelian
			and finite).
			\begin{itemize}
				\item injective:
					$\phi(a) = \phi(b) \Leftrightarrow a^n = b^n \Leftrightarrow 
						(ab^{-1})^n = e \Leftrightarrow ord(ab^{-1}) \mid n$.
						We also have $ord(ab^{-1}) \mid \vert G \vert$, so $ord(ab^{-1}) \mid
						gcd(n,\vert G \vert) = 1$, then $ord(ab^{-1}) = 1$, which means $a =
						b$.
				\item surjective:
					Beacuse $G$ is finite and $\phi$ is injective, $\vert G \vert = \vert
					\phi(G) \vert$. We also have $\phi(G) \subseteq G$. Then we have
					$\phi(G) = G$.
				\item homomorphic:
					$\phi(ab) = (ab)^n = a^nb^n = \phi(a)\phi(b)$
			\end{itemize}
			Above all, $\phi$ is an automorphism of $G$.
	\section*{Exercise 2-2}
		\subsection*{2}
			$C(G)$ is not empty beacuse $e \in C(G)$. For all $a, b \in C(G)$ and $x \in G$, 
			we have:
			$$
				ab^{-1}x = a(x^{-1}b)^{-1} = a(bx^{-1})^{-1} = axb^{-1} = xab^{-1}
			$$
			So $ab^{-1} \in C(G)$ and $C(G) \preceq G$.

			For all $g \in G$, $gC(G) = C(G)g$(beacuse $gx = xg$ holds for all $x \in
			C(G)$), so $C(G)$ is a normal subgroup of $G$.
		\subsection*{5}
			Let $G = S_4, K = \{ (1), (12)(34), (13)(24), (14)(23) \}, H = \{ (1),
			(12)(34)\}$.
		\subsection*{6}
			\begin{itemize}
				\item $\Rightarrow$:
					If $ab \in H$ and $H \unlhd G$, we assume $ab = h \in H$ and then
					$$
						ba = babb^{-1} = bhb^{-1} \in H
					$$
				\item $\Leftarrow$:
					For every $h \in H$ and $g \in G$, we have $(hg^{-1})g \in H$, then 
					$g(hg^{-1}) = ghg^{-1} \in H$. So $H \unlhd G$.
			\end{itemize}
		\subsection*{9}
			For all $a, b \in N(H)$, we have $ aHa^{-1} = H $ and $bHb^{-1} = H$, then
			\begin{equation*}
				\begin{split}
					(ab^{-1})H(ab^{-1})^{-1} &= ab^{-1}Hba^{-1} = a(bH^{-1}b^{-1})^{-1}a^{-1} \\
					& = a(bHb^{-1})^{-1}a^{-1} = aH^{-1}a^{-1} = aHa^{-1} = H
				\end{split}
			\end{equation*}
			So $ab^{-1} \in N(H)$, which means $N(H) \preceq G$.

			For all $n \in N(H)$, we have
			$$
				nHn^{-1} = H
			$$
			which means $H \unlhd N(H)$.

		\subsection*{10}
			\begin{itemize}
				\item $\Rightarrow$:
					Let $\phi(x) = gxg^{-1}$ be an inner automorphism of $G$. For all $h
					\in H$, we have $\phi(h) = ghg^{-1} \in H$(beacuse $H$ is a normal
					subgroup of $G$), which means $\phi(H) \subseteq H$.
				\item $\Leftarrow$:
					For every $g \in G$, it can generate an inner automorphism $\phi(x) =
					gxg^{-1}$. Beacuse $\phi(H) \subseteq H$, we have $gHg^{-1} \subseteq
					H$. Beacuse for all $g \in G$ we have $gHg^{-1} \subseteq H$, then $H$
					is a normal subgroup of $G$.
			\end{itemize}
		\subsection*{11}
			For every $x \in G$, we can get a left coset of $H$ generated by $x$ : $xH$.
			Beacuse 
			$$
			\vert G / H \vert = [G : H] = m
			$$
			we have
			$$
			(xH)^m = x^mH^m = x^mH = eH
			$$
			which means $x^m \in H$.
	\section*{Exercise 2-3}
		\subsection*{1}
			$x \rightarrow \vert x \vert$ and $x \rightarrow x^2$ are two
			homomorphisms. When $a = 1$, $x \rightarrow ax$ is also a homomorphism.
			\begin{itemize}
				\item $\phi:x \rightarrow \vert x \vert$:
					$\phi(G) = R^{+}$ and $Ker(\phi) = \{1, -1\}$
				\item $\phi:x \rightarrow x$:
					$\phi(G) = G$ and $Ker(\phi) = \{1\}$
				\item $\phi:x \rightarrow x^2$:
					$\phi(G) = R^{+}$ and $Ker(\phi) = \{1, -1\}$
			\end{itemize}
		\subsection*{6}
			Let's call that map $\phi$. For every $x,y \in \mathbb{C}^{*}$, we have
			$$
				\phi(xy) = (xy)^6 = x^6y^6 = \phi(x)\phi(y)
			$$
			so $\phi$ is a homomorphism.
			
			Solve the following equation
			$$
				\phi(x) = x^6 = 1
			$$
			we can get $6$ solutions:$\{ 1, w, w^2, w^3, w^4, w^5\} (w =
			e^{\frac{i\pi}{3}})$.And the kernal of $\phi$ is
			$$
			Ker(\phi) = \{ 1, w, w^2, w^3, w^4, w^5\} (w = e^{\frac{i\pi}{3}})
			$$
		\subsection*{7}
			For every $q \in \{0, 1, 2, \cdots, m - 1\}$,  we can define a homomorphism:
			$$
				\phi_q(n) = qn \bmod m
			$$
		\subsection*{16}
			\begin{itemize}
				\item $\Rightarrow$:
					\begin{equation*}
						\begin{split}
							\phi(a) = \phi(b) & \Rightarrow \phi(ab^{-1}) = e' \Rightarrow
							ab^{-1} \in Ker(\phi) \\
							& \Rightarrow Ker(\phi)a = Ker(\phi)b \Rightarrow aKer(\phi) = bKer(\phi)
						\end{split}
					\end{equation*}
					(The last steps use the conclusion that $Ker(\phi)$ is a normal
					subgroup of $G$).
				\item $\Leftarrow$:
					\begin{equation*}
						\begin{split}
							aKer(\phi) = bKer(\phi) &\Rightarrow \phi(aKer(\phi))=\phi(bKer(\phi)) \\
							 & \Rightarrow \phi(a)e' = \phi(b)e' \Rightarrow \phi(a) = \phi(b)
						\end{split}
					\end{equation*}
			\end{itemize}
		\subsection*{18}
			\begin{itemize}
				\item $\phi^{-1}(\phi(H)) \subseteq HK$: For any $x \in
					\phi^{-1}(\phi(H))$, there exists $h \in H$ such $\phi(x) = \phi(h)$,
					then
					$$
					\phi(xh^{-1}) = e' \Rightarrow xh^{-1} \in K \Rightarrow xh^{-1}h \in
					Kh \Rightarrow x \in Kh \subseteq KH \Rightarrow x \in HK
					$$
				\item $HK \subseteq \phi^{-1}(\phi(H))$: For any $x \in HK$, there exist
					$h \in H$ and $k \in K$ such $x = hk$, then
					$$\phi(x) = \phi(hk) = \phi(h)$$
					which means $x \in \phi^{-1}(\phi(h))$, thus $x \in \phi^{-1}(\phi(H))$.
			\end{itemize}
		\subsection*{19}
			\begin{itemize}
				\item $\Rightarrow$:
					Let $\phi:G_1\rightarrow G_2$ be the epimorphism. There must exist $a \in
					G_1$ such that $\phi(a) = 1'$. We have $n_2 = ord(1') \mid ord(a)$ and
					$ord(a) \mid n_1$, thus $n_2 \mid n_1$.
				\item $\Leftarrow$:
					We can construct an epimorphism $\phi(x) = x \bmod n_2$.
					\begin{itemize}
						\item surjective:
							For any $x \in G'$, $x$ must also belong $G$(Beacuse $n_2 \leq
							n_1$).
						\item homomorphic:
							\begin{equation*}
							\begin{split}
								\phi(a+b) &= ((a+b) \bmod n_1) \bmod n_2 = (a + b) \bmod n_2 \\
								          &= ((a \bmod n_2) + (b \bmod n_2)) \bmod n_2 = \phi(a) + \phi(b)
							\end{split}
							\end{equation*}
							(Beacuse $n_2 \mid n_1$)
					\end{itemize}
					So $G_1 \sim G_2$.
			\end{itemize}
		\subsection*{12}
			We know that $Z_5 \cong U_5$ by the isomorphism $f(x) = e^{\frac{2x\pi}{5}}$, so we only
			need to consider $\phi:Z_{30} \sim Z_5$.

			Let $\phi(1) = a$, we have $ord(a) \mid 5$ and $ord(a) \mid ord(1) = 30$.
			So $ord(a) = 1$ or $ord(a) = 5$. Beacuse $\phi$ is surjective, so $ord(a)
			= 5$, thus $a$ can be one of $\{1,2,3,4\}$ and $a$ satisfy $gcd(a,5) = 1$.
			So $\phi(x) = xa \equiv 0 \Leftrightarrow x \equiv 0 \pmod 5$.
			
			So the kernel of $\phi$ is:
			$$
				Ker(\phi) = \{ 0, 5, 10, 15, 20, 25 \}
			$$
			
		\subsection*{20}
			We define $\phi:Z_m \rightarrow Z_k(x \rightarrow x \bmod k)$.
			Beacuse $m \geq k$, we can see that $\phi$ is surjective.
			$$\phi(a+b) = ((a+b) \bmod m) \bmod k = (a \bmod k) + (b \bmod k) \bmod k =
			\phi(a) + \phi(b)$$
			Thus $\phi$ is homomorphic. So $\phi$ is an epimorphism.
			$$
			\phi(x) = 0 \Leftrightarrow x \bmod k = 0 \Leftrightarrow k \mid x
			$$
			So we can get the kernel of $\phi$:
			$$
				Ker(\phi) = \{ x \in Z_m \mid k \mid x \} = \{ 0, k, 2k, \dots,
				m - k\} = \langle k \rangle
			$$
			By the First Isomorphism Theorem, we have
			$$
			Z_m/\langle k \rangle \cong Z_k
			$$
	\section*{Exercise 2-5}
		\subsection*{1}
			\begin{equation*} 
				\begin{split}
					X &= \{ 1, 2, 3, 4, 5, 6, 7, 8 \}	\\
					G &= \{ (1), (123)(456), (132)(465), (78), (123)(456)(78), (132)(465)(78)\}
				\end{split}
			\end{equation*}
			\begin{center}
				\begin{tabular}{r | c | c}
					\hline
						x & Orbit & Stabilizer \\
					\hline 
						$ 1 $ & $\{ 1, 2, 3 \}$ & $\{ (1), (78) \}$ \\
						$ 2 $ & $\{ 1, 2, 3 \}$ & $\{ (1), (78) \}$ \\
						$ 3 $ & $\{ 1, 2, 3 \}$ & $\{ (1), (78) \}$ \\
						$ 4 $ & $\{ 4, 5, 6 \}$ & $\{ (1), (78) \}$ \\
						$ 5 $ & $\{ 4, 5, 6 \}$ & $\{ (1), (78) \}$ \\
						$ 6 $ & $\{ 4, 5, 6 \}$ & $\{ (1), (78) \}$ \\
						$ 7 $ & $\{ 7, 8 \}$ & $\{ (1), (123)(456), (132)(465) \}$ \\
						$ 8 $ & $\{ 7, 8 \}$ & $\{ (1), (123)(456), (132)(465) \}$ \\
					\hline
				\end{tabular}
			\end{center}
			\begin{center}
				\begin{tabular}{r | c}
					\hline
					g & Fixed elements \\
					\hline
					$(1)           $ & $\{1,2,3,4,5,6,7,8\}$\\
					$(123)(456)    $ & $\{7,8\}$\\
					$(132)(465)    $ & $\{7,8\}$\\
					$(78)          $ & $\{1,2,3,4,5,6\}$\\
					$(123)(456)(78)$ & $\varnothing$\\
					$(132)(465)(78)$ & $\varnothing$\\
					\hline
				\end{tabular}
			\end{center}
		\subsection*{3}
			Beacuse the action of $G$ on $X$ is transitive, for any $x_1, x_2 \in X$,
			exist $g \in G$ such that
			$$
				gx_1 = x_2
			$$
			Then for any $x_1, x_2 \in X$, let's consider their orbits
			$$
				gx_1 = x_2 \Rightarrow Ngx_1 = Nx_2 \Rightarrow gNx_1 = Nx_2
			$$
			Beacuse the action $g$ on $X$ is injective, so we have
			$$
				\vert Nx_1 \vert = \vert g(Nx_1) \vert = \vert Nx_2 \vert
			$$
			So every orbit has the same size.
		\subsection*{4}
			\begin{itemize}
				\item $gS_xg^{-1}\subseteq S_y$:
					For any $h \in S_x$, we have $hx = x$. Then
					$$
					ghg^{-1}y = ghx = gx = y
					$$
					so $ghg^{-1} \in S_y$.
				\item $S_y\subseteq gS_xg^{-1}$:
					For any $h \in S_y$, we have $hy = y$. Then
					$$
					g^{-1}hgx = g^{-1}hy = g^{-1}y = x
					$$
					so $g^{-1}hg \in S_x$. Thus $g^{-1}S_yg \subseteq S_x$, which means
					$S_y \subseteq gS_xg^{-1}$.
			\end{itemize}
			So $S_y = gS_xg^{-1}$.

		\subsection*{6}
			Let the $12$ faces be 
			$X = \{ \pi_1, \pi_2, \pi_3, \pi_4, \pi_5, \pi_6, \pi_7, \pi_8, \}$ and
			$\pi_1$ be the front face.
			We have 
			\begin{equation*}
				\begin{split}
					O_{\pi_1} & = X \\
					\vert S_{\pi_1} \vert & = 5
				\end{split}
			\end{equation*}
			So $\vert G \vert = \vert O_{\pi_1} \vert \vert S_{\pi_1}\vert = 40$.
		\subsection*{Additional}
			It's the same as Exercisse 2-5(4).
	\section*{Exercise 2-6}
		\subsection*{1}
			\begin{itemize}
				\item $\Rightarrow$:
					Beacuse there is only one Sylow p-subgroup $P$, thus for all $g \in G$ we
					have
					$$
						gPg^{-1} = P,
					$$
					which means $P$ is a normal subgroup of $G$.
				\item $\Leftarrow$:
					By the Sylow Theorem, when $P$ is a Sylow p-subgroup, all the Sylow
					p-subgroup will be
					$$
						\{ gPg^{-1} \mid g \in G\}.
					$$
					Beacuse $P$ is a normal subgroup, $gPg^{-1} = P$ for all $g \in G$,
					which means there is only one Sylow p-subgroup $P$.
			\end{itemize}
		\subsection*{2}
			\begin{itemize}
				\item $N(P) \subseteq N(N(P))$:
					For every $g \in N(P)$, 
					$$
						gN(P)g^{-1} = N(P) \Rightarrow g \in N(N(P))
					$$.
				\item $N(N(P)) \subseteq N(P)$:
					For every $g \in N(N(P))$, we have
					$$
						gPg^{-1} \subseteq gN(P)g^{-1} = N(P).
					$$
					So $gPg^{-1}$ is another Sylow p-subgroup of $N(P)$, thus 
					$$
						gPg^{-1} = P \Rightarrow g \in N(P)
					$$
					Then we have $N(N(P)) \subseteq N(P)$.
			\end{itemize}
		\subsection*{3}
			The Sylow 2-subgroup of $S_4$:
			\begin{equation*}
				\begin{split}
					\{(1),(1234),(13)(24),(1432),(13),(12)(34),(24),(14)(23)\}, \\
					\{(1),(1324),(12)(34),(1423),(12),(13)(24),(34),(14)(32)\}, \\
					\{(1),(1243),(14)(23),(1342),(14),(12)(43),(23),(13)(24)\},
				\end{split}
			\end{equation*}
		\subsection*{4}
			The Sylow 2-subgroup of $A_4$:
			$$
				\{(1),(14)(23),(13)(24),(14)(23)\},
			$$
		\subsection*{6}
			By the Sylow Theorem
			$$
				n_5 \mid 24 \quad \text{and} \quad n_5 \equiv 1 \pmod 5,
			$$
			there are two posible values $n_5 = 1$ or $n_5 = 6$. But there are more
			than one Sylow 5-subgroup in $S_5$, so $n_5 = 6$.
			\begin{equation*}
				\begin{split}
					\{(1),(12345),(13524),(14253),(15432)\} \\
					\{(1),(13452),(14235),(15324),(12543)\}
				\end{split}
			\end{equation*}

\end{document}

