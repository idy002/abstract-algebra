\documentclass{article}
\usepackage{amsmath}
\usepackage{amssymb}
\usepackage{titlesec}

\title{Homework}
\date{2017-09-22}
\author{Ding Yaoyao, 516030910572}

\begin{document}

	\maketitle

	\section*{Exercise 1-3}
		\subsection*{5}

		\begin{itemize}
			\item Closure: If $a,b \in G$, there are $c,d \in G$ such that $a = c^m, b
				= d^m$. Then $ab=c^md^m=(cd)^m \in G$(Beacuse group $G$ is an abelian
				group).
			\item Identity: $e = e^m \in G$.
			\item Inverse: If $a = b^m \in G$, then $a^{-1} = (b^m)^{-1} = (b^{-1})^m
				\in G$.
		\end{itemize}

		Above all, $H \preccurlyeq G$.

		\subsection{6}

		For all $a, b \in gHg^{-1}$, there are $c,d \in H$ such that $a = gcg^{-1},
		b = gdg^{-1}$.

		Then $a^{-1}b = (gcg^{-1})^{-1}gdg^{-1} = gc^{-1}g^{-1}gdg^{-1} =
		gc^{-1}dg^{-1}$.

		Beacuse $c^{-1}d \in H$, we have $a^{-1}b \in gHg^{-1}$. 

		Finally, we get $gHg^{-1} \preccurlyeq G$.

		\subsection*{7}	

		For all $b, c \in C(a)$, we have $ba = ab$ and $ca = ac$.

		Then $b^{-1}c = (aba^{-1})^{-1}aca^{-1} = ab^{-1}a^{-1}aca^{-1} =
		ab^{-1}ca^{-1}$, which imply that $ab^{-1}c=b^{-1}ca$, $b^{-1} \in C(a)$.

		Above all, $C(a) \preccurlyeq G$.

		\subsection*{8}
			$g \in C(G) \Leftrightarrow \forall a \in G, ga=ag \Leftrightarrow \forall
			a \in G, g \in C(a) \Leftrightarrow g \in \bigcap_{a \in G}C(a)$, then we
			have:
			$$
				C(G) = \bigcap_{a \in G}C(a)
			$$

		\subsection*{18}

		\begin{itemize}
			\item $<m,n> \subseteq <d>$: $a \in <m,n> \Rightarrow \exists k_1, k_2, a
				= k_1m+k_2n = (k_1\frac{m}{d} + k_2\frac{n}{d})d \Rightarrow a \in <d>$.
		  \item $<d> \subseteq <m,n>$: By the Euclid theorem, $\exists k_1, k_2, d =
				k_1m+k_2n$. If $a \in <d>$, we have $a = kd = k(k_1m + k_2n) = (kk_1)m +
				(kk_2)n$, which means $a \in <m,n>$.
		\end{itemize}
		
		Above all, $<m,n> = <d>$.

		\subsection*{19}

		It's obvious that $m = \pm n$ imply $<m> = <n>$, we only need to prove $<m>
		= <n> \Rightarrow m = \pm n$.

		If $<m> = <n>$, we have $m \in <m> = <n>$, which means $n \mid m$. We can
		get $m \mid n$ by the same method. But $n \mid m$ and $m \mid n$ means $m =
		\pm n$. That's all.

\end{document}

