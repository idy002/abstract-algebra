\documentclass[draft]{article}
\usepackage{amsmath}
\usepackage{amssymb}
\usepackage{titlesec}
\usepackage{booktabs}

\title{Homework}
\date{2017-11-24}
\author{Ding Yaoyao, 516030910572}

\begin{document}

	\maketitle
	\section*{Exercise 2-4}
		\subsection*{1}
			If $(a,b) \in Z_9 \oplus Z_6$ and $\vert (a,b) \vert = 9$, we have 
			$$
				lcm(\vert a \vert, \vert b \vert) = 9.
			$$
			Also $\vert a \vert$ may be one of $\{ 1, 3, 9\}$ and $\vert b \vert$ may be
			one of $\{ 1, 2, 3, 6\}$. Thus $\vert a \vert = 9$ and $\vert b \vert = 1 \;
			\text{or} \; 3$.  Thus there are totally $6 \times 3 = 18$ elements whose order is $9$.
		\subsection*{3}
			$Z\oplus Z$ is not a cyclic group.

			Proof by contradiction:

			If $Z \oplus Z$ is a cyclic group, we let $(a,b)$ be a generator. Then we
			can get $\vert a \vert = 1$ and $\vert b \vert = 1$(otherwise $(1,1)$ can
			not be generated by $(a,b)$,here $\vert a \vert$ means the absolute value of $a$).

			But it's obvious that any one of $(1,1), (1,-1), (-1,1), (-1,-1)$ can not 
			generate $Z \oplus Z$(for example, none can generate $(1,0)$).

			So $Z \oplus Z$ is not a cyclic group.
		\subsection*{5}
			Let $a = (1,0) \in Z_8 \oplus Z_2$, we have $\vert a \vert = 8$.
			
			If $(b,c) \in Z_4 \oplus Z_4$, we have $\vert (b,c) \vert = lcm(\vert b
			\vert, \vert c \vert) = max(\vert b \vert, \vert c \vert) \leq 4$.

			So $Z_8 \oplus Z_2$ is not isomorpic to $Z_4 \oplus Z_4$.
		\subsection*{7}
			Let $S = \{ 1, -1 \}$.
			\begin{itemize}
				\item $S \unlhd R^*$ and $R^+ \unlhd R^*$:
					Beacuse the group $R^*$ is an abelian group, all of its subgroup are
					normal.
				\item $R^* = R^+S$:
					By $R^+ \preceq R^*$ and $S \preceq R^*$, we can get $R^+S \subseteq
					R^*$.

					For any $x \in R^*$, we have 
					$$
						x = \vert x \vert\phi(x) 
					$$
					$\phi(x) = 1$ if $x > 0$ and $\phi(x) = -1$ if $x < 0$.
					Beacuse $\phi(x) \in \{ -1, 1\}$ and $\vert x \vert \in Z^+$, we have $x \in
					R^+S$. Then we can get $R^* \subseteq R^+S$.

					Thus $R^* = R^+S$.


				\item $R^+ \cap S = \{1\}$: $-1 \not\in R^+$ and $1 \in R^+$, so $R^+
					\cap S = \{1\}$.
			\end{itemize}
			Thus $R^*$ is the internal direct product of $R^+$ and $\{1,-1\}$.
		\subsection*{8}

		Beacuse $\vert (1,0) + H \vert = 2$,  $\vert (0,1) + H \vert = 2$ and 
		$(1,0) + H \neq (0,1) + H$, we know that $G/H \cong Z_2 \oplus Z_2$.
		
		Beacuse $\vert (1,1) + K \vert = 4$, we know that $G/K \cong Z_4$.

		\subsection*{14}
			Let $a = ord(a_1,a_2,\dots,a_n)$ and $b =
			lcm(ord(a_1),ord(a_2),\cdots,ord(a_n))$.
			\begin{itemize}
				\item $a \mid b$:
					$$(a_1,a_2,\dots,a_n)^b = (a_{1}^b,a_{2}^b,\dots,a_n^b) = (1,1,\cdots,1)$$
					Then $a \mid b$.
				\item $b \mid a$:
					$$
						(a_1,a_2,\dots,a_n)^a = (a_1^a,a_2^a,\dots,a_n^a) = (1,1,\cdots,1)
					$$
					Then
					$$
						a_i^a = 1 \Rightarrow ord(a_i) \mid a (\text{for all} i = 1, 2,
						\dots, n)
					$$
					So
					$$
						b = lcm(ord(a_1),ord(a_2),\dots,ord(a_n)) \mid a
					$$
					Thus $b \mid a$.
			\end{itemize}
			So $ord(a_1,a_2,\dots,a_n) = lcm(ord(a_1),ord(a_2),\dots,ord(a_n))$.
		\subsection*{Addition}
			\textbf{Problem:Show that $Z_{145}$ is a cyclic group.}

			By Sylow Theorem, there are Sylow 29-subgroup and Sylow 5-subgroup.
			Beacuse
			$$
				n_{29} \equiv 1 \pmod{29} \text{ and } n_{29} \mid 5
			$$
			We have $n_{29} = 1$. Similarly, we can get $n_5 = 1$.
			So there are only one Sylow 29-subgroup and one Sylow 5-subgroup, we call
			them $H$ and $K$, respectively. Then $H$ and $K$ are both normal
			subgroups.

			If $a \in H\cap K$, $\vert a \vert \mid 29$ and $\vert a \vert \mid 5$,
			then $\vert a \vert = 1$, thus $H\cap K = \{1\}$.

			$$\vert HK \vert = \frac{\vert H \vert \vert K \vert}{\vert H \cap K \vert}
			= 145.$$
			So $HK = Z_{145}$.

			So $G = HK$ is an internal direct product of $H$ and $K$, Thus $G \cong H
			\times K$.

			The order of $(1,1) \in H \times K$ is $lcm(29,5) = 145$, so $H\times K$
			is a cylic group, thus $G$ is a cylic group.



\end{document}
