\documentclass{article}
\usepackage{amsmath}

\title{Homework}
\date{2017-10-13}
\author{Ding Yaoyao, 516030910572}

\begin{document}

	\maketitle

	\section*{Exercise 1-6}
		\subsection*{5}
			\subsubsection*{(1)}
				$\tau \sigma \tau^{-1} = (1)$
			\subsubsection*{(2)}
				$\tau \sigma \tau^{-1} = (1 \; 3 \; 4 \; 2)$
		\subsection*{12}
			There are totally 6 subgroups.
			\begin{itemize}
				\item $\{(1)\}$.
				\item $\{(1 \; 2), (1)\}.$
				\item $\{(1 \; 3), (1)\}.$
				\item $\{(2 \; 3), (1)\}.$
				\item $\{(1 \; 2 \; 3), (1 \; 3 \; 2), (1)\}.$
				\item $\{(1 \; 2 \; 3), (1 \; 3 \; 2), (1 \; 2), (1 \; 3), (2 \; 3), (1)\}.$
			\end{itemize}
		\subsection*{24}
			Let $G = S_n$. Beacuse odd permutation exists in $G$, we have $n \geq 2$.
			Then we get $(1 \; 2) \in G$. We define:
			\begin{align*}
				A & = \{ \tau \in G \mid \text{ $\tau$ is a even permutation} \} \\
				B & = \{ \tau \in G \mid \text{ $\tau$ is an odd permutation} \} 
			\end{align*}

			If $\tau \in A$, by the definition of even/odd permutation, we have $(1 \;
			2) \in B$.

			We define a mapping:$f: A \rightarrow B$ and $f(\tau) = (1 \; 2)\tau$ for
			all $\tau \in A$.

			If $f(\tau_1) = f(\tau_1)$, we have $(1 \; 2)\tau_1 = (1 \; 2)\tau_2$. We
			can get $\tau_1 = \tau_2$ by the elimination law of group. So $f$ is
			injective.

			For all $\sigma \in B$, we have $\tau = (2 \; 1)\sigma \in A$ such that
			$f(\tau) = (1 \; 2)(2 \; 1)\sigma = \sigma$. So $f$ is surjective.
			
			Beacuse $f$ is a bijection between $A$ and $B$, we have $\mid A \mid =
			\mid B \mid$.
		\subsection*{25}
			Beacuse $(1) \in H$, $H$ is not empty.

			For all $\tau, \sigma \in H$, we have:
			\begin{align*}
				\tau & = (a_1a_2)(a_3a_4)\cdots(a_{r-1}a_r) \\
				\sigma & = (b_1b_2)(b_3b_4)\cdots(b_{s-1}b_s) 
			\end{align*}
			($r$ and $s$ are even numbers,maybe zero.)

			Then we have $\tau^{-1}\sigma =
			(a_ra_{r-1})\cdots(a_4a_3)(a_2a_1)(b_1b_2)(b_3b_4)\cdots(b_{s-1}b_s)$ and
			$r + s$ is a even number. By the definition of even permutation, we have $\tau^{-1}\sigma \in H$.

			Above all, $H \subseteq G$.
\end{document}
