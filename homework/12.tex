\documentclass[draft]{article}
\usepackage{amsmath}
\usepackage{amssymb}
\usepackage{titlesec}
\usepackage{booktabs}

\title{Homework}
\date{2017-12-01}
\author{Ding Yaoyao, 516030910572}

\begin{document}

	\maketitle
	\section*{Exercise 3-1}
		\subsection*{1}
			\subsubsection*{(1)}
				Beacuse the operation $\oplus$ does not satisfy $(a\oplus b)\oplus c = a
				\oplus (b \oplus c)$, $S$ does not form a ring.
			\subsubsection*{(2)}
				Beacuse the operations $\oplus$ and $*$ do not satisfy $(a\oplus b)*c =
				(a*c)\oplus(b*c)$, $S$ does not form a ring.
			\subsubsection*{(3)}
				$S$ does not form a ring by the same reason of $(2)$.
		\subsection*{4}
			Let $u, v, w \in Z[\sqrt{3}]$ and $u = a_1 + b_1\sqrt{3}$, $v =
			a_2+b_2\sqrt{3}$, $w = a_3 + b_3\sqrt{3}$.	
			\begin{itemize}
				\item closure:	
					$u + v = (a_1+b_1\sqrt{3}) + (a_2+b_2\sqrt{3})$,$ =
					(a_1+a_2)+(b_1+b_2)\sqrt{3} \in Z[\sqrt{3}]$.
				\item associativity:
					$(u + v) + w = (a_1 + a_2 + a_3) + (b_1 + b_2 + b_3)\sqrt{3} = u + (v
					+ w)$.
				\item identity: 
					$0 + u = u + 0 = u$, so $0$ is an identity.
				\item inverse:
					Let $u^{-1} = (-a_1) + (-b_1)\sqrt{3}$, then $uu^{-1} = u^{-1}u = 0$.
				\item abelian:
					$u+v = (a_1+b_1\sqrt{3}) + (a_2+b_2\sqrt{3}) = v+u$.
			\end{itemize}
			Then $(Z[\sqrt{3}],+)$ is an abelian group.
			\begin{itemize}
				\item closure:
					$uv = (a_1a_2+3b_1b_2) + (a_1b_2+a_2b_1)\sqrt{3} \in Z[\sqrt{3}]$.
				\item associativity:
					$(uv)w =
					(a_1a_2a_3+3a_1b_2b_3+3a_2b_1b_2+3a_3b_1b_2)+(b_1a_2a_3+b_2a_1a_3+b_3a_1a_2)\sqrt{3}
					= u(vw)$.
				\item identity:
					$1u = u1 = u$, so $1$ is an identity.
				\item abelian:
					$uv = vu$.
			\end{itemize}
			Then $(Z[\sqrt{3}],\cdot)$ is an abelian monoid.
			Also
			\begin{equation*}
				\begin{split}
					(u+v)w & = [(a_1+a_2) + (b_1+b_2)\sqrt{3}](a_3+b_3\sqrt{3}) \\
					       & = (a_1+b_1\sqrt{3})(a_3+b_3\sqrt{3}) +
								 (b_1+b_2\sqrt{3})(a_3+b_3\sqrt{3}) \\
								 & = uw+vw
				\end{split}
			\end{equation*}
			Similarly, $w(u+v) = wu + wv$.
			So $+$ and $\cdot$ satisfy distributive laws.

			Above all, $(Z[\sqrt{3},+,\cdot)$ is an abelian ring with identity.
		\subsection*{17}
			$(2)(3)(4)$.
		\subsection*{18}
			$(1)(4)$.
	\section*{Exercise 3-2}
		\subsection*{2}
			Let $u, v, w \in Z[\theta]$ and $u = a_1+b_1\theta$,$v = a_2+b_2\theta$,$w
			= a_3+b_3\theta$.
			\begin{itemize}
				\item closure:	
					$u + v = (a_1+a_2)+(b_1+b_2)\theta \in Z[\theta]$.
				\item associativity: 
					$(u+v)+w = u+(v+w)$(For all $u,v,w\in \mathbb{C}$).
				\item identity:
					$0 \in Z[\theta]$ and $0+u = u+0 = u$, so $0$ is the identity.
				\item inverse:
					Let $(-u) = (-a_1) + (-b_1)\theta \in Z[\theta]$, then $u+(-u) =
					(-u)+u = 0$.
				\item abelian:
					$uv=vu$(For all $u,v\in \mathbb{C}$).
			\end{itemize}
			Then $(Z[\theta],+)$ is an abelian group.
			\begin{itemize}
				\item closure:
					$uv = (a_1a_2-b_1b_2)+(a_1b_2+a_2b_1+b_1b_2)\theta \in Z[\theta]$.
				\item associativity:
					$(uv)w=u(vw)$(For all $u,v \in \mathbb{C}$).
				\item identity:
					Let $e = 1 \in Z[\theta]$ and $e$ satisfy $eu=ue=u$, so $e$ is the
					identity.
				\item abelian:
					$uv=vu$(For all $u,v \in \mathbb{C}$).
				\item no zeros:
					If $u \neq 0$ and $v \neq 0$, then $uv \neq 0$(For all $u,v\in
					\mathbb{C}$).
			\end{itemize}
			$(u+v)w = uw+vw$ and $w(u+v) = wu+wv$ are true for all $u,v \in
			\mathbb{C}$.
	
			Above all, $(Z[\theta],+,\cdot)$ is an integral domain.

			If $u$ is a unit, then
			$$
			\frac{1}{u} = \frac{1}{a+b\theta} = \frac{a+b-b\theta}{a^2+b^2+ab}
			$$
			then we have
			$$
			a^2 + b^2 + ab \mid a+b
			$$
			Then we have $u = 1, -1, \theta, -\theta, 1-\theta, \theta-1$.
		\subsection*{9}
			The zeros:
			$$
			\{ (a,b,c) \mid abc = 0 \text{ and } (a,b,c) \neq (0,0,0) \}
			$$
			The units:
			$$
			\{ (a,b,c) \mid \vert a \vert = \vert c \vert = 1, b \neq 0 \}
			$$
	\section*{Supplement}
		If $I$ is an ideal of ring $\mathbb{Z}_N$, $I$ must also be a subgroup of
		$\mathbb{Z}_N$ under the $+$ operation. Thus $I = d\mathbb{Z}_N$, where $d =
		0$ or $d \mid N$. 
		
		For any $d = 0$ or $d \mid N$, we can get an subring $I = d\mathbb{Z}_N$.
		For any $a \in \mathbb{Z}_N$, we have $aI = (ad)\mathbb{Z}_N \subseteq
		d\mathbb{Z}_N$, thus subring $I$ is an ideal of $\mathbb{Z}_N$.
\end{document}
