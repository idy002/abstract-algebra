\documentclass[draft]{article}
\usepackage{amsmath}
\usepackage{amssymb}
\usepackage{titlesec}
\usepackage{booktabs}

\title{Homework}
\date{2017-12-08}
\author{Ding Yaoyao, 516030910572}

\begin{document}

	\maketitle
	\section*{Exercise 3-3}
		\subsection*{9}
			$$
				I = \{ (a-2b) + (b+2a)i \mid a, b \in \mathbb{Z} \}
			$$
			$$
				\mathbb{Z}[i]/I = \{ I, i+I, 2i+I, 3i+I, 4i+I \}
			$$
		\subsection*{13}
			Let $z_1, z_2 \in IJ$, it's obvious that $z_1 - z_2 \in IJ$, thus $IJ$ is a
			subgroup of $R$ on $+$. If $z = \sum x_iy_i \in IJ$ and $t \in R$, then 
			$$
			tz = \sum (tx_i)y_i \in IJ \quad \text{ and } \quad zt = \sum x_i(y_it) \in IJ,
			$$
			thus $IJ$ is an ideal of $R$.

			If $z = \sum x_iy_i \in IJ$, then $z = \sum x_iy_i = \sum (x_i)' \in I$
			and $z = \sum x_iy_i = \sum (y_i)' \in J$, then $z \in I\cap J$, thus $IJ
			\subset I \cap J$.
		\subsection*{17}
			\subsubsection*{(1)}
				$\langle 3 \rangle$.
			\subsubsection*{(2)}
				$\langle 30 \rangle$.
			\subsubsection*{(3)}
				$\langle 90 \rangle$.
	\section*{Exercise 3-4}
		\subsection*{5}
			If $f$ is the homomorphism from $\mathbb{Z}_4$ to $\mathbb{Z}_{20}$, then
			we have 
			$$
			f(0) = 0 \quad \text{and} \quad f(1) = 1
			$$
			Beacuse $f(a+b) = f(a) + f(b)$, we have $f(n) = n$ for $n = 0, 1, 2, 3$.
			But $f(3+3) = f(3) + f(3) = 6$ and $f(3+3) = f(2) = 2$, so $f$ is not a
			homomorphism. Thus the homomorphism does not exist.

		\subsection*{7}
			\subsubsection*{(2)}
				Similarly with above, there is only one homomorphism:
				
				$f(n) = n \quad (n = 0, 1, \dots, 9)$.
			\subsubsection*{(3)}
				There is only one homomorphism:$f(n) = n \quad (n = 0, 1, \dots, 11)$.
\end{document}
