\documentclass{article}
\usepackage{amsmath}
\usepackage{amssymb}
\usepackage{titlesec}
\usepackage{booktabs}

\title{Homework}
\date{2017-09-29}
\author{Ding Yaoyao, 516030910572}

\begin{document}

	\maketitle

	\section*{Exercise 1-4}
		\subsection*{Additional}

		 Problem: If $N = n_1 n_2$, and $gcd(n_1, n_2) = 1$, then $Z_n^{*} \cong Z_{n_1}^{*}\times Z_{n_2}^{*}$.

		 Proof:

		\begin{itemize}
			\item Define the map $\varphi$: $Z_n^{*} \rightarrow Z_{n_1}^{*} \times
				Z_{n_2}^{*}$ by $\varphi(x) = (x \; mod \; n_1, x \; mod \; n_2)$.
			\item $\varphi$ is injective: If $(x \; mod \; n_1, x \; mod \; n_2) = (y
				\; mod \; n_1, y \; mod \; n_2)$, then $n_1 \mid x - y$ and $n_2 \mid x
				- y$. Beacuse $n_1$ and $n_2$ are co-prime, $n_1n_2 \mid x - y$, which
				means $n \mid x - y$. Thus $x = y$(when $x,y \in [0,n)$).
			\item $\varphi$ is surjective: For all $(x_1,x_2) \in Z_{n_1}^*\times
				Z_{n_2}^{*}$, exist $x \in Z_n^{*}$ such that $x \; mod \; n_1 = x_1$
				and $x \; mod \; n_2 = x_2$ by the Chinese Remainder Theorem(when
				$gcd(n_1,n_2) = 1$).
			\item $\varphi$ is homomorphic: 
				$\varphi(xy) = (xy \; mod \; n_1, xy \; mod \; n_2) $ 

				$= ((x \; mod \; n_1)(y \; mod \; n_1) \; mod \; n_1, (x \; mod \; n_2)(y \; mod \; n_2) \; mod \; n_2) $ 

				$ = (x \; mod \; n_1, x \; mod \; n_2)(y \; mod \; n_1, y \; mod \; n_2) $ 

				$= \varphi(x)\varphi(y)$.
		\end{itemize}

		Above all, $Z_n^{*} \cong Z_{n_1}^{*}\times Z_{n_2}^{*}$.

		\subsection*{3}
			\begin{itemize}
				\item $\Rightarrow$: For all $x \in G$, exist $y \in G$ such that
					$y^{-1} = x$. So $\phi$ is surjective. If $\phi(x) = \phi(y)$(i.e.
					$x^{-1} = y^{-1}$), we have $x = y$. So $\phi$ is injective. Beacuse
					group $G$ is an abelian group, we have:
					$$
					\phi(xy) = (xy)^{-1} = y^{-1}x^{-1} = x^{-1}y^{-1} = \phi(x)\phi(y)
					$$
					Then $\phi$ is homomorphic. Above all, $\phi$ is an
					isomorphism(exactly automorphism).
				\item $\Leftarrow$: Beacuse $\phi$ is an isomorphism, we have:
					$$
					xy = ((xy)^{-1})^{-1} = \phi((xy)^{-1}) = \phi(y^{-1}x^{-1}) =
					\phi(y^{-1})\phi(x^{-1}) = yx
					$$
					(for all $x,y \in G$). So group $G$ is an abelian group.
			\end{itemize}
		\subsection*{4}
			\begin{itemize}
				\item injective: If $\phi(x) = \phi(y)$, then $axa^{-1} = aya^{-1}$. By
					the cancellation law of group, we get $x = y$.
				\item surjective: For all $x \in G$, exist $y = a^{-1}xa \in G$ such that
					$aya^{-1} = x$.
				\item homomorphic: $\phi(xy) = a^{-1}xya = a^{-1}x(aa^{-1})ya =
					(a^{-1}xa)(a^{-1}ya) = \phi(x)\phi(y)$.
			\end{itemize}
			Above all, $\phi$ is an isomorphism(also called inner automorphism).
		\subsection*{6}
			$H = \langle 1 \rangle, G = \langle 2 \rangle, R = \langle 4
			\rangle$($\langle k \rangle$ means group ($\{kn \mid n \in \mathbb{Z}\},+)$).

			It's obvious that $\langle 4 \rangle < \langle 2 \rangle < \langle 1
			\rangle$ and $\langle 2 \rangle \cong \langle 2 \rangle$(by the identity
			isomorphism). And I only need to prove $\langle 1 \rangle \cong \langle 4 \rangle$.

			Let's define the mapping $\phi$: $\langle 1 \rangle \rightarrow \langle 4
			\rangle$ as $\phi(x) = 4x$ for all $x \in \mathbb{Z}$.

			\begin{itemize}
				\item injective: If $\phi(x) = \phi(y)$, then $4x = 4y$. We can get $x =
					y$.
				\item surjective: For all $x \in \langle 4 \rangle$, by the definition
					of $\langle 4 \rangle$, exist $y \in \mathbb{Z}$ such that $4y =
					x$(i.e. $\phi(y) = x$).
				\item homomorphic: $phi(x+y) = 4(x+y) = 4x + 4y = \phi(x) + \phi(y)$.
			\end{itemize}

			Above all, $\phi$ is an isomorphism, which means $\langle 1 \rangle \cong
			\langle 4 \rangle$.

	\section*{Exercise 1-5}
		\subsection*{1}
			\begin{table}[!htp]
				\centering
				\caption{$Z_{7}$}
				\begin{tabular}{c|*{7}{c}}
					\toprule
					$n$ & $0$ & $1$ & $2$ & $3$ & $4$ & $5$ & $6$ \\ \midrule
					$ord$ & $1$ & $7$ & $7$ & $7$ & $7$ & $7$ & $7$ \\ \bottomrule
				\end{tabular}
			\end{table}
			\begin{table}[!htp]
				\centering
				\caption{$Z_{8}$}
				\begin{tabular}{c|*{8}{c}}
					\toprule
					$n$ & $0$ & $1$ & $2$ & $3$ & $4$ & $5$ & $6$ & $7$ \\ \midrule
					$ord$ & $1$ & $8$ & $4$ & $8$ & $2$ & $8$ & $4$ & $8$ \\ \bottomrule
				\end{tabular}
			\end{table}
			\begin{table}[!htp]
				\centering
				\caption{$Z_{10}$}
				\begin{tabular}{c|*{10}{c}}
					\toprule
					$n$ & $0$ & $1$ & $2$ & $3$ & $4$ & $5$ & $6$ & $7$ & $8$ & $9$ \\ \midrule
					$ord$ & $1$ & $10$ & $5$ & $10$ & $5$ & $2$ & $5$ & $10$ & $5$ & $10$ \\ \bottomrule
				\end{tabular}
			\end{table}
			\begin{table}[!htp]
				\centering
				\caption{$Z_{14}$}
				\begin{tabular}{c|*{14}{c}}
					\toprule
					$n$ & $0$ & $1$ & $2$ & $3$ & $4$ & $5$ & $6$ & $7$ & $8$ & $9$ & $10$ & $11$ & $12$ & $13$ \\ \midrule
					$ord$ & $1$ & $14$ & $7$ & $14$ & $7$ & $14$ & $7$ & $2$ & $7$ & $14$ & $7$ & $14$ & $7$ & $14$ \\ \bottomrule
				\end{tabular}
			\end{table}
			\begin{table}[!htp]
				\centering
				\caption{$Z_{15}$}
				\begin{tabular}{c|*{15}{c}}
					\toprule
					$n$ & $0$ & $1$ & $2$ & $3$ & $4$ & $5$ & $6$ & $7$ & $8$ & $9$ & $10$ & $11$ & $12$ & $13$ & $14$ \\ \midrule
					$ord$ & $1$ & $15$ & $15$ & $5$ & $15$ & $3$ & $5$ & $15$ & $15$ & $5$ & $3$ & $15$ & $5$ & $15$ & $15$ \\ \bottomrule
				\end{tabular}
			\end{table}
			\begin{table}[!htp]
				\centering
				\caption{$Z_{18}$}
				\begin{tabular}{c|*{18}{c}}
					\toprule
					$n$ & $0$ & $1$ & $2$ & $3$ & $4$ & $5$ & $6$ & $7$ & $8$ & $9$ & $10$ & $11$ & $12$ & $13$ & $14$ & $15$ & $16$ & $17$ \\ \midrule
					$ord$ & $1$ & $18$ & $9$ & $6$ & $9$ & $18$ & $3$ & $18$ & $9$ & $2$ & $9$ & $18$ & $3$ & $18$ & $9$ & $6$ & $9$ & $18$ \\ \bottomrule
				\end{tabular}
			\end{table}

		\subsection*{5}
	
			$U(n)$ is an cyclic group for all $n \in \mathbb{Z}^{+}$.

			The generators:

			\begin{itemize}
				\item $U(8)$:$e^{2\pi i\frac{1}{8}}$, $e^{2\pi i\frac{3}{8}}$, $e^{2\pi i\frac{5}{8}}$, $e^{2\pi i\frac{7}{8}}$
				\item $U(9)$:$e^{2\pi i\frac{1}{9}}$, $e^{2\pi i\frac{2}{9}}$, $e^{2\pi i\frac{4}{9}}$, $e^{2\pi i\frac{5}{9}}$, $e^{2\pi i\frac{7}{9}}$, $e^{2\pi i\frac{8}{9}}$
				\item $U(10)$:$e^{2\pi i\frac{1}{10}}$, $e^{2\pi i\frac{3}{10}}$, $e^{2\pi i\frac{7}{10}}$, $e^{2\pi i\frac{9}{10}}$
				\item $U(13)$:$e^{2\pi i\frac{1}{13}}$, $e^{2\pi i\frac{2}{13}}$, $e^{2\pi i\frac{3}{13}}$, $e^{2\pi i\frac{4}{13}}$, $e^{2\pi i\frac{5}{13}}$, $e^{2\pi i\frac{6}{13}}$, $e^{2\pi i\frac{7}{13}}$, $e^{2\pi i\frac{8}{13}}$, $e^{2\pi i\frac{9}{13}}$, $e^{2\pi i\frac{10}{13}}$, $e^{2\pi i\frac{11}{13}}$, $e^{2\pi i\frac{12}{13}}$
				\item $U(14)$:$e^{2\pi i\frac{1}{14}}$, $e^{2\pi i\frac{3}{14}}$, $e^{2\pi i\frac{5}{14}}$, $e^{2\pi i\frac{9}{14}}$, $e^{2\pi i\frac{11}{14}}$, $e^{2\pi i\frac{13}{14}}$
				\item $U(21)$:$e^{2\pi i\frac{1}{21}}$, $e^{2\pi i\frac{2}{21}}$, $e^{2\pi i\frac{4}{21}}$, $e^{2\pi i\frac{5}{21}}$, $e^{2\pi i\frac{8}{21}}$, $e^{2\pi i\frac{10}{21}}$, $e^{2\pi i\frac{11}{21}}$, $e^{2\pi i\frac{13}{21}}$, $e^{2\pi i\frac{16}{21}}$, $e^{2\pi i\frac{17}{21}}$, $e^{2\pi i\frac{19}{21}}$, $e^{2\pi i\frac{20}{21}}$
			\end{itemize}

		\subsection*{12}
			We only need to prove: 
			$$
				(gag^{-1})^r = e \Longleftrightarrow a^r = e
			$$
			\begin{itemize}
				\item $\Rightarrow$: If $(gag^{-1})^r = ga^rg^{-1} = e$, we have $ga^r =
					g$, which means $a^r = e$.
				\item $\Leftarrow$: $(gag^{-1})^r = ga^rg^{-1} = gg^{-1} = e$.
			\end{itemize}

\end{document}

